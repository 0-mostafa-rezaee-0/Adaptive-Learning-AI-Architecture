% =============================================================================
% AI ASSISTANT PROMPT - READ THIS CAREFULLY
% =============================================================================
% 
% IMPORTANT: When using this template, AI assistants should follow these guidelines:
% 
% 1. STRUCTURE PRESERVATION:
%    - Maintain the exact section hierarchy: \section{} -> \subsection{} -> \subsubsection{}
%    - Keep all \label{} commands for cross-references
%    - Preserve the appendix structure with \appendix command
% 
% 2. CONTENT FORMATTING:
%    - Use placeholder text format: "Placeholder [description]" for empty sections
%    - Replace placeholders with actual content while maintaining structure
%    - Keep mathematical equations in proper \begin{equation} environments
%    - Use \begin{itemize} and \begin{enumerate} for lists
%    - Maintain table and figure environments with proper \label{} and \ref{}
% 
% 3. TECHNICAL REQUIREMENTS:
%    - Keep all package imports at the top
%    - Maintain custom commands: \milestone{}, \metric{}, \risk{}, \mitigation{}
%    - Preserve hyperref configuration for black TOC links
%    - Keep page styling and header/footer settings
% 
% 4. CONTENT GUIDELINES:
%    - Use \textbf{} for bold text, \textit{} for italic, \texttt{} for code
%    - Use \ref{label} for cross-references to sections, tables, figures
%    - Use \cite{} for citations (add bibliography if needed)
%    - Maintain consistent formatting throughout
% 
% 5. TEMPLATE VARIABLES:
%    - Replace PROJECT NAME, PROJECT SUBTITLE, AUTHOR TEAM with actual values
%    - Update document metadata as needed
%    - Keep the template structure intact for reusability
% 
% 6. QUALITY STANDARDS:
%    - Ensure all sections have meaningful content (not just placeholders)
%    - Maintain academic writing standards
%    - Use proper LaTeX syntax and avoid errors
%    - Test compilation before finalizing
% 
% FOLLOW THIS TEMPLATE EXACTLY - DO NOT MODIFY THE STRUCTURE
% =============================================================================

% !TEX program = pdflatex
% !TEX encoding = UTF-8 Unicode
% LaTeX Template - Single File Version
% Version: 1.0
% Author: Template Creator
% License: MIT

\documentclass[11pt,letterpaper]{article}

% =============================================================================
% PACKAGE IMPORTS
% =============================================================================

% Core LaTeX packages
\usepackage[utf8]{inputenc}
\usepackage[T1]{fontenc}
\usepackage[margin=1in]{geometry}
\usepackage{graphicx}
\usepackage{amsmath}
\usepackage{amssymb}
\usepackage{amsfonts}

% Table and formatting packages
\usepackage{booktabs}
\usepackage{longtable}
\usepackage{array}
\usepackage{enumitem}
\usepackage{multirow}
\usepackage{colortbl}

% Hyperlinks and references
\usepackage[hidelinks]{hyperref}
\usepackage{cleveref}

% Color and graphics
\usepackage{xcolor}

% Code listings
\usepackage{listings}
\usepackage{fancyhdr}
\usepackage{float}
\usepackage{afterpage}
\usepackage{placeins}

% =============================================================================
% DOCUMENT METADATA
% =============================================================================

% Template variables (customize these for each document)
\newcommand{\projectname}{PROJECT NAME}
\newcommand{\projectsubtitle}{PROJECT SUBTITLE}
\newcommand{\authorteam}{AUTHOR TEAM}
\newcommand{\documentversion}{1.0}
\newcommand{\documentdate}{\today}
\newcommand{\documentstatus}{Draft} % Draft, Review, Final

% =============================================================================
% PAGE STYLE AND FORMATTING
% =============================================================================

% Fix header height issue
\setlength{\headheight}{13.6pt}
\addtolength{\topmargin}{-1.6pt}

\pagestyle{fancy}
\fancyhf{}
\rhead{\projectname}
\lhead{\authorteam}
\rfoot{Page \thepage}
\fancypagestyle{plain}{\fancyhf{}\rfoot{Page \thepage}}

% =============================================================================
% TITLE AND AUTHOR INFORMATION
% =============================================================================

\title{\textbf{\projectname:}\\\projectsubtitle}
\author{\authorteam}
\date{\documentdate}

% =============================================================================
% CUSTOM COMMANDS AND ENVIRONMENTS
% =============================================================================

% Milestone and metric formatting
\newcommand{\milestone}[1]{\textbf{\#1}}
\newcommand{\metric}[1]{\texttt{\#1}}

% Risk and mitigation formatting
\newcommand{\risk}[1]{\textcolor{red}{\textbf{Risk:} #1}}
\newcommand{\mitigation}[1]{\textcolor{blue}{\textbf{Mitigation:} #1}}

% Technical term highlighting
\newcommand{\techterm}[1]{\textit{#1}}
\newcommand{\code}[1]{\texttt{#1}}

% Custom environments
\newenvironment{techspec}
{\begin{quote}\textbf{Technical Specification:}}
{\end{quote}}

\newenvironment{implementation}
{\begin{quote}\textbf{Implementation Note:}}
{\end{quote}}

% =============================================================================
% HYPERREF CONFIGURATION
% =============================================================================

\hypersetup{
    colorlinks=true,
    linkcolor=black,
    filecolor=black,      
    urlcolor=cyan,
    citecolor=green,
    bookmarksnumbered=true,
    bookmarksopen=true,
    pdfstartview=FitH
}

% =============================================================================
% DOCUMENT CONTENT
% =============================================================================

\begin{document}

\maketitle

\begin{abstract}
This is a placeholder abstract. Replace this text with a brief summary of your document. The abstract should provide a concise overview of the main points, methodology, and conclusions of your work.
\end{abstract}

\tableofcontents
\newpage

% =============================================================================
% MAIN SECTIONS
% =============================================================================

\section{Introduction}\label{sec:introduction}

This is a placeholder introduction section. Replace this text with your actual introduction content. This section should provide background information, motivation, and an overview of what the document covers.

\subsection{Background}\label{subsec:background}

Placeholder background information goes here. This subsection can contain historical context, related work, or foundational concepts that readers need to understand before diving into the main content.

\subsection{Objectives}\label{subsec:objectives}

Placeholder objectives section. Clearly state what you aim to achieve in this document. Use bullet points or numbered lists to make objectives clear and measurable.

\begin{itemize}
\item Placeholder objective 1
\item Placeholder objective 2
\item Placeholder objective 3
\end{itemize}

\section{Methodology}\label{sec:methodology}

This is a placeholder methodology section. Describe your approach, tools, and techniques used in your work.

\subsection{Data Collection}\label{subsec:data-collection}

Placeholder data collection methodology. Describe how you gathered your data, what sources you used, and any limitations or constraints.

\subsection{Analysis Framework}\label{subsec:analysis-framework}

Placeholder analysis framework. Explain your analytical approach, including any mathematical models, algorithms, or frameworks you used.

\section{Mathematical Formulations}\label{sec:mathematical-formulations}

This section demonstrates mathematical notation and equations. Replace with your actual mathematical content.

\subsection{Basic Equations}\label{subsec:basic-equations}

Here are some placeholder mathematical equations:

\begin{equation}
E = mc^2
\end{equation}

\begin{equation}
f(x) = \int_{-\infty}^{\infty} e^{-x^2} dx = \sqrt{\pi}
\end{equation}

\begin{align}
\frac{\partial f}{\partial x} &= \lim_{h \to 0} \frac{f(x+h) - f(x)}{h} \\
\frac{\partial f}{\partial y} &= \lim_{h \to 0} \frac{f(y+h) - f(y)}{h}
\end{align}

\subsection{Matrix Operations}\label{subsec:matrix-operations}

Matrix notation examples:

\begin{equation}
A = \begin{bmatrix}
a_{11} & a_{12} & a_{13} \\
a_{21} & a_{22} & a_{23} \\
a_{31} & a_{32} & a_{33}
\end{bmatrix}
\end{equation}

\begin{equation}
Ax = b \quad \text{where} \quad x = A^{-1}b
\end{equation}

\section{Tables and Figures}\label{sec:tables-figures}

This section demonstrates how to include tables and figures in your document.

\subsection{Sample Table}\label{subsec:sample-table}

Table \ref{tab:sample-table} shows a placeholder data table.

\begin{table}[H]
\centering
\caption{Sample Data Table}
\label{tab:sample-table}
\begin{tabular}{|l|c|r|}
\hline
\textbf{Category} & \textbf{Value} & \textbf{Percentage} \\
\hline
Placeholder A & 100 & 25\% \\
Placeholder B & 200 & 50\% \\
Placeholder C & 100 & 25\% \\
\hline
\textbf{Total} & \textbf{400} & \textbf{100\%} \\
\hline
\end{tabular}
\end{table}

\subsection{Sample Figure}\label{subsec:sample-figure}

Figure \ref{fig:sample-figure} shows a placeholder figure.

\begin{figure}[H]
\centering
\fbox{\parbox{6cm}{\centering
\vspace{2cm}
\textbf{Placeholder Figure}\\
\vspace{0.5cm}
This is a placeholder for your figure.\\
Replace this with your actual figure using\\
\texttt{\textbackslash includegraphics\{filename\}}
\vspace{2cm}
}}
\caption{Sample Placeholder Figure}
\label{fig:sample-figure}
\end{figure}

\section{Results and Analysis}\label{sec:results-analysis}

This is a placeholder results section. Present your findings, data analysis, and interpretations here.

\subsection{Key Findings}\label{subsec:key-findings}

Placeholder key findings. Highlight the most important results from your work:

\begin{itemize}
\item Placeholder finding 1 with supporting evidence
\item Placeholder finding 2 with statistical significance
\item Placeholder finding 3 with practical implications
\end{itemize}

\subsection{Statistical Analysis}\label{subsec:statistical-analysis}

Placeholder statistical analysis. Include any statistical tests, confidence intervals, or significance tests you performed.

\section{Discussion}\label{sec:discussion}

This is a placeholder discussion section. Interpret your results, discuss implications, and relate your findings to existing literature.

\subsection{Implications}\label{subsec:implications}

Placeholder implications discussion. What do your results mean for the field, practice, or future research?

\subsection{Limitations}\label{subsec:limitations}

Placeholder limitations section. Be honest about the constraints and limitations of your work:

\begin{itemize}
\item Placeholder limitation 1
\item Placeholder limitation 2
\item Placeholder limitation 3
\end{itemize}

\section{Conclusion}\label{sec:conclusion}

This is a placeholder conclusion section. Summarize your main findings, contributions, and suggest future work.

\subsection{Summary}\label{subsec:summary}

Placeholder summary of main contributions and findings.

\subsection{Future Work}\label{subsec:future-work}

Placeholder future work suggestions:

\begin{enumerate}
\item Placeholder future work item 1
\item Placeholder future work item 2
\item Placeholder future work item 3
\end{enumerate}

% =============================================================================
% APPENDICES
% =============================================================================

\appendix

\section{Technical Specifications}\label{app:technical-specs}

This is a placeholder appendix for technical specifications.

\subsection{System Requirements}\label{app:system-requirements}

Placeholder system requirements:

\begin{itemize}
\item Placeholder requirement 1
\item Placeholder requirement 2
\item Placeholder requirement 3
\end{itemize}

\subsection{Implementation Details}\label{app:implementation-details}

Placeholder implementation details and technical specifications.

\section{Additional Data}\label{app:additional-data}

This is a placeholder appendix for additional data, tables, or supplementary information.

% =============================================================================
% DOCUMENT END
% =============================================================================

\end{document}
